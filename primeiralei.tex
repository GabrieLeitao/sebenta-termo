
\begin{theorem}[Lei Zero da Termodinâmica]
    Quando dois objetos estão em equilíbrio térmico com um terceiro objeto, os dois estão em equilíbrio térmico entre si.
\end{theorem}

\section{Balanço de Energia}

A partir das contribuições de Galileu e outros, Newton formulou as leis do movimento, fundamentais na mecânica.

Considerando a força tangencial ao movimento e $\mathbf{v} = \frac{ds}{dt}$:

\begin{eqnarray*}
    F = m \frac{d\mathbf{v}}{dt} = m \frac{d\mathbf{v}}{ds} \frac{ds}{dt} = m \mathbf{v} \frac{d\mathbf{v}}{ds} \\
    \int_{\mathbf{v}_1}^{\mathbf{v}_2} m \mathbf{v} \, d\mathbf{v} = \int_{s_1}^{s_2} F \, ds \\
    \Delta E_c = \frac{1}{2} m \left( \mathbf{v}_2^2 - \mathbf{v}_1^2 \right) = \int_{s_1}^{s_2} F \, ds
\end{eqnarray*}

Ou seja, o trabalho de uma força ao longo da trajetória equivale à variação da energia cinética.

Seja $R$ a força resultante exceto a gravítica, com $g$ constante:

\begin{eqnarray*}
    \frac{1}{2} m \left( \mathbf{v}_2^2 - \mathbf{v}_1^2 \right) = \int_{s_1}^{s_2} R \, ds - \int_{z_1}^{z_2} mg \, dz \\
    \Delta E_p = mg(z_2 - z_1) \\
    \int_{s_1}^{s_2} R \, ds = \Delta E_c + \Delta E_p
\end{eqnarray*}

As energias cinética e potencial são propriedades extensivas do sistema, com unidades de energia, iguais às do trabalho: em SI, sendo $W = \int \vec{F} \cdot d\vec{s}$, o trabalho tem unidades $\text{N} \cdot \text{m}$, que se denomina Joule, J.

O \textbf{trabalho} é uma forma de transferir energia. O trabalho $W$ depende nos detalhes das interações a ocorrer entre um sistema e a vizinhança num determinado processo, pelo que o trabalho não é uma propriedade. O diferencial de trabalho, $\delta W$, é inexato: $\int_1^2 \delta W = W$. Por outro lado, o diferencial de uma propriedade é exato. Por exemplo, para o volume: $\int_{V_1}^{V_2} dV = V_2 - V_1$.

A taxa com que dada transferência de energia ocorre, fruto de trabalho, denomina-se \textbf{potência}, $\dot{W}$, unidade SI é o Watt, W.

\begin{eqnarray*}
    \dot{W} = \frac{d}{dt} \int \vec{F} \cdot d\vec{s} = \vec{F} \cdot \vec{\mathbf{v}}\\
    W = \int \dot{W} dt
\end{eqnarray*}

\subsection{Trabalho de Compressão/Expansão}

Neste texto, convenciona-se que qualquer forma de energia que entra no sistema é positiva e energia que sai é negativa, \textbf{seguindo a convenção nas aulas, contrária à do Shapiro}.

Considerando um sistema êmbolo-cilindro, a pressão do gás exerce uma força no pistão, durante a expansão. O trabalho feito pelo sistema quando o êmbolo se move é:  

\begin{equation*}
    \delta W = - F \, dx = - pA \, dx \implies \delta W = - p \,dV
\end{equation*}

Assim, o trabalho é dado por 

\begin{equation}
    W = - \int_{V_1}^{V_2} p \, dV
\end{equation}

Desta forma, tal como convencionado: durante uma expansão $dV > 0$ e $W < 0$, o sistema perde energia; enquanto numa compressão $dV < 0$ e $W > 0$, o sistema recebe energia.

A equação do trabalho é válida para sistemas de qualquer forma, desde que a pressão seja uniforme ao longo da fronteira móvel.

O módulo do trabalho corresponde à área sob a curva num diagrama $p$-$V$. Como diferentes processos entre os mesmos estados 1 e 2 podem gerar curvas distintas, o trabalho depende do processo realizado.

\subsubsection{Processo Politrópico}

Um processo de quase equilíbrio descrito por $pV^n = \text{const.}$ ou $pv^n = \text{const.}$, onde $n$ é uma constante, chama-se \textbf{processo politrópico}. 

O trabalho para um processo politrópico, onde $c = p V^n = p_1 V_1^n = p_2 V_2^n$ e $p = \frac{c}{V^n}$, é:

\begin{itemize}
    \item Para $n \neq 1$:
    \begin{equation*}
        \begin{split}
            W & = - \int_{V_1}^{V_2} \frac{c}{V^n} \, dV  = - \frac{c V_2^{1-n} - c V_1^{1-n}}{1-n} = \frac{(p_2 V_2^n) V_2^{1-n} - (p_1 V_1^n) V_1^{1-n}}{n - 1} \\
            W & = \frac{p_2 V_2 - p_1 V_1}{n - 1}, \quad n \neq 1
        \end{split}
    \end{equation*}
    \item Para $n = 1$:
    \begin{equation*}
        \begin{split}
            W & = - \int_{V_1}^{V_2} \frac{c}{V} \, dV  = - c \ln \frac{V_2}{V_1} \\
            W & = - p_1 V_1 \ln \frac{V_2}{V_1} = - p_2 V_2 \ln \frac{V_2}{V_1}, \quad n = 1
        \end{split}
    \end{equation*} 
\end{itemize}

Dado que $p_1V_1^n = p_2V_2^n$:

\begin{eqnarray}
    \frac{p_1}{p_2} = \left( \frac{V_2}{V_1} \right)^n \Longleftrightarrow 
    \ln \left( \frac{V_2}{V_1} \right)^n = \ln \frac{p_1}{p_2} \Longleftrightarrow 
    n \ln \frac{V_2}{V_1} = \ln \frac{p_1}{p_2} \implies
    n = \frac{\ln \frac{p_1}{p_2}}{\ln \frac{V_2}{V_1}}
\end{eqnarray}

\subsection{Energia Interna}

A variação da energia total de um sistema é calculada com base em três grandes contribuições: energia cinética, energia potencial -- associadas ao movimento e posição do sistema --, e \textbf{energia interna}, $U$, onde estão concentradas as outras contribuições. Todas estas energias são propriedades extensivas.
\begin{equation}
    \Delta E = \Delta U + \Delta E_c + \Delta E_p 
\end{equation}

Microscopicamente, a energia interna inclui: energia cinética translacional, rotacional e vibracional das moléculas; energia das ligações químicas; energia atómica (orbitais, spin) e nuclear. Em substâncias condensadas, as forças intermoleculares também contribuem significativamente para $U$.

\subsection{Calor}

Da mesma forma que o trabalho, o \textbf{calor} não é uma propriedade, pois depende do processo. $Q = \int_1^2 \delta Q$.
A taxa de transferência de energia por calor pode ser dada por:

\begin{equation*}
    \dot{Q} = \int_A \dot{q} \, dA
\end{equation*}

onde $\dot{q}$ é o fluxo de calor por unidade de área.

A transferência de calor pode ser por \textbf{condução} ou \textbf{radiação}, e uma combinação que produz a \textbf{convecção}.

\subsubsection{Condução}

A condução é a transferência de energia de moléculas mais energéticas de uma substância para partículas adjacentes.

A Lei de Fourier permite calcular a taxa de transferência de energia por condução:

\begin{equation}
    \dot{Q}_x = - \kappa A \frac{dT}{dx}, \quad \dot{\vec{Q}} = - \kappa A \nabla T
\end{equation}

onde $\kappa$ é uma constante chamada condutividade térmica. O sinal negativo reflete o facto da transferência ocorrer na direção de menor temperatura.

\subsubsection{Radiação}

A radiação transmite energia por ondas eletromagnéticas, fotões. Ao contrário da condução, a radiação pode transferir energia no vácuo.

A taxa de transferência de energia por radiação pode ser calculada por uma forma modificada da Lei de Stefan-Boltzmann:

\begin{equation}
    \dot{Q}_e = \epsilon \sigma A T^4
\end{equation}

onde $T$ é a temperatura da superfície, $\epsilon$ é a emissividade da superfície ($0 \leq \epsilon \leq 1$), e $\sigma = 5.67 \cdot 10^{-8} \; \text{W}/\text{m}^2 \cdot \text{K}^4$.

A emissividade $\epsilon$ é uma grandeza adimensional que quantifica a capacidade de uma superfície emitir radiação térmica em comparação com um corpo negro ideal ($\epsilon = 1$). Superfícies metálicas polidas, por serem altamente refletoras, têm baixa emissividade, enquanto superfícies negras ou rugosas apresentam valores próximos de $1$.

A emissividade depende do material, do estado da superfície, da temperatura e do comprimento de onda. Em muitos casos, utiliza-se uma emissividade média ou espetral para simplificação.


\subsubsection{Convecção}

A lei de arrefecimento de Newton dá-nos uma expressão experimental para calcular a transferência de energia por convecção de uma superfície sólida à temperatura de $T_b$ e um gás ou líquido adjacente a $T_f$.
\begin{equation}
    \dot{Q}_c = h A (T_b - T_f), \quad \dot{Q} = h A \Delta T
\end{equation}

onde $h$ é um coeficiente de transferência de calor, determinado experimentalmente, que depende do fluido, da geometria, da rugosidade da superfície e das condições de escoamento (como velocidade e regime laminar ou turbulento).


\subsection{Primeira Lei da Termodinâmica}

Joule demonstrou experimentalmente que a energia é conservada, i.e. a energia não pode ser criada ou destruída, apenas convertida. Isto é considerado a Primeira Lei da Termodinâmica.

Para um sistema fechado, Joule deduziu que o trabalho em qualquer processo adiabático entre dois estados de equilíbrio é o mesmo. Como depende apenas dos estados inicial e final, por definição de propriedade, esse trabalho corresponde à variação de uma propriedade do sistema, a energia, i.e., \(\Delta E = W_{\text{adia}}\).

Do \textbf{balanço de energia} geral resulta:

\begin{equation}
    \Delta E = \Delta E_c + \Delta E_p + \Delta U = Q + W
\end{equation}

Na forma diferencial:

\begin{equation}
    dE = \delta Q + \delta W, \qquad \frac{dE}{dt} = \frac{dE_c}{dt} + \frac{dE_p}{dt} + \frac{dU}{dt} = \dot{Q} + \dot{W}
\end{equation}

Recorrentemente, as energias cinéticas e potenciais macroscópicas do sistema são desprezáveis:

\begin{equation}
    \Delta U = Q + W, \qquad \frac{dU}{dt} = \dot{Q} + \dot{W}
\end{equation}


\section{Volumes de Controlo}

Para sistemas com fluxo de massa, existe transferência de energia associado ao trabalho do fluxo volumétrico que entra ou sai:

\begin{equation*}
    \frac{dE}{dt} = \vec{F} \cdot \vec{\mathbf{v}} = p A \mathbf{v} = p A \frac{dx}{dt} = p \frac{dV}{dt} = p \frac{v dm}{dt} = \dot{m} pv 
\end{equation*}

e o balanço energético para o volume de controlo é:

\begin{equation}
    \frac{dE}{dt} = \frac{dU}{dt} + \frac{dE_c}{dt} + \frac{dE_p}{dt} = \dot{Q} + \dot{W} + \sum_i \dot{m} \left( \frac{1}{2}\mathbf{v}_i^2 + gz_i + u_i + p_i v_i \right)
\end{equation}

Deste modo, define-se a \textbf{entalpia} como $h = u + pv$, uma grandeza útil para o estudo. Desprezando termos cinéticos e potenciais:

\begin{equation}
    \frac{dU}{dt} = \dot{Q} + \dot{W} + \sum_i \dot{m} h_i
\end{equation}

Muitas vezes a entalpia é muito maior comparativamente aos termos de energia cinética e potencial associada ao fluxo de massa que passa a fronteira do sistema ($h_i \gg \frac{1}{2}\mathbf{v}_i^2 + gz_i$), sendo, por isso, desprezáveis.